\section{Introduction}

\paragraph{}The assessment of breast cancer screening programs traditionally has been based on indicators of performance, e.g. frequency of detected cancers, false-positive results, and interval cancers, obtained cross-sectionally at specific time periods (usually a screening round). Cross-sectional measures are limited, however, by the fact that they do not capture changes over time that may be associated to the risk of developing breast cancer. In contrast, longitudinal studies of women attending breast cancer screening may assess how past and current information on risk factors affect the risk of developing breast cancer.

\paragraph{}In research on breast cancer screening, longitudinal data methods were initially applied to estimate the cumulative rates of false-positive results \cite{Elmore1998, Castells2006, Hubbard2011, Roman2011b, Hofvind2012d}. Later on, longitudinal data analyses have been used to study 1) the frequency of screen-detected cancers over several rounds \cite{Hofvind2006, Blanch2013}, 2) the effect of previous false-positive results on detection rate \cite{VonEuler-Chelpin2012, Castells2013b}, 3) the determinants of interval cancer \cite{Hofvind2006, Blanch2014}, and 4) the association between longitudinal breast density measurements and breast cancer risk \cite{Armero2016}.
Interval cancers are symptomatic cancers that appear between two screening exams or during a period after the last screening exam.

\paragraph{}In a previous work, we studied the determinants of screen-detected cancer and interval cancer  -symptomatic breast cancer that appears after negative tests and before the next invitation \cite{Blanch2014}. We analyzed screen-detected and interval cancers as distinct causes of failure, but, the fact breast cancer is detected in a screening exam precludes an interval cancer, and vice versa. Progression of breast cancer fits better in the multi-state models theory were individuals transition among different health states at variable rates that can be related to their characteristics. 

\paragraph{}Screening for breast cancer may lead to interventions in asymptomatic women, resulting in data on the natural disease course that are observed partially and selective. As a consequence, screening may cause several biases in the analysis of time to event data. In analysis of survival from diagnosis, the most known and studied biases are \textit{lead-time} and \textit{length} sampling. The lead-time is defined as the time gained by diagnosing the disease before the patient experiences symptoms. Even if early diagnosis and early treatment had no benefit, the survival of early detected cancer cases would be longer than the survival of clinical cases. Length sampling bias arises because screen-detected cancers are more likely to have slower growth than non-screen detected cancers \cite{Zelen1969}. In addition, when screen detected and symptomatic cancers are combined, screening may interfere in the assessment of risk factors of breast cancer due to 1) earlier time of detection for screen detected tumours; 2) the fact that symptomatically detected tumours have characteristics that are different from screen detected cancers \cite{Blanch2014}; and 3) there is overdiagnosis of low growth tumours that never would become symptomatic during a woman lifetime \cite{Esserman2013a}. In summary, early detection of breast cancer introduces changes in the natural history of the disease that need to be considered when assessing the effect of interventions or the association of risk factors.

\paragraph{}Simulation models can be used to examine how screening interferes with the assessment of risk factors. Recently, Taghipour \textit{et al.} used a partially observable Markov model to estimate relevant parameters of breast cancer progression and early detection in the Canadian National Breast Screening Study \cite{Taghipour2013}. Previously, other authors had modeled the natural history of breast cancer, using analytic or simulation models that incorporated input data from the literature and made different assumptions \cite{Zelen1969,Chen1996,Shen2001,Lee2003b,Berry2006,Fryback2006,Hanin2006,Lee2006,Mandelblatt2006, Plevritis2006,Weedon-Fekjaer2008a}. The work of the Cancer Intervention and Surveillance Modeling network (CISNET) breast group \cite{Berry2006,Fryback2006,Hanin2006,Lee2006,Mandelblatt2006, Plevritis2006} reflects how modeling studies can provide evidence to complement the results of randomized controlled trials and observational studies. 

\paragraph{}The objectives of this study are 1) To estimate the biases induced by the screening process when assessing the effect of a false positive result in a mammogram test on the progression to pre-clinical cancer and clinical cancer; 2) To evaluate the effect estimates obtained via parametric multi-state models that take interval censoring into account and semi-parametric Cox models or discrete-time models that don't.The paper is organized as follows. Section 2 presents the motivating study, the INCA-CAT,  which contains data on sequential screening mammograms in Catalan women. Section 3 is a methods section that defines concepts relevant to screening, describes the methods used to estimate the effect of false-positive results in previous mammographic exams on the hazard of being diagnosed of breast cancer, and also details how a simulation study was performed. Section 4 presents the results of the simulation study and their application to the INCA-CAT study. Section 5 contains the discussion and conclusions.



