\section{Motivating data: the INterval CAncer (INCA) study}
The aim of the INCA study was to assess the determinants of interval breast cancer in women
attending early detection programs in Spain \cite{Domingo2014, Blanch2014}. The INCA study also compared the characteristics associated with screen-detected and interval breast cancers. We selected a subset of the INCA dataset, corresponding to four radiology units of the Catalonia region, the INCA-CAT dataset. 

\paragraph{}Population-based breast cancer screening in Catalonia is offered biennially to all
women aged 50-69 years. Screening mammography has three possible outcomes: 1) negative result
(normal), 2) positive result (abnormal findings requiring further assessments), and 3) early recall
(an intermediate mammogram is performed out of sequence with the screening interval, at 6 or 12
months). Cancers detected at an intermediate mammogram were considered screen-detected cancers \cite{Perry2008}.

\paragraph{}A positive result is considered to be a screen-detected tumour if, after further
assessments, there is histopathological confirmation of cancer. Otherwise, the result is considered
false-positive and the woman is invited again after two years. Further assessments can include
non-invasive procedures (magnetic resonance imaging, ultrasonography, additional mammography) and/or
invasive (fine-needle aspiration cytology, core-needle biopsy and open biopsy). 

\paragraph{}Interval cancer was defined as ``a primary breast cancer arising after a negative
screening episode and before the next invitation to screening or within 24 months for women who
reached the upper age limit" \cite{Perry2008}. This definition was extended until the 30th month, to allow a 6 months margin for women to attend each round. Interval cancers were identified by merging the screening programmes databases with hospital discharge databases and population-based or hospital cancer registries. 

\paragraph{}The INCA-CAT dataset consists of a retrospective cohort of 96,636 women who underwent
mammography exams between January 1, 2000 and December 31, 2006, and were followed-up until June
30, 2009 for interval cancer assessment. These women underwent a total of 230,742 screening
mammograms. During the study period, 963 cancers were diagnosed, of which 671 were detected in screening exams, and 313 emerged as interval cancers. Both invasive and ductal \textit{in situ} (non-invasive) breast cancer carcinomas are included. 

\paragraph{} Among several risk factors studied, the existence of a false positive result in the previous mammogram showed the highest hazard ratio for developing interval cancer, HR=2.71 \cite{Blanch2014}. The corresponding hazard ratio for screen-detected cancer was 1.34. The authors also reported that the effect of a previous false positive result was higher for interval cancers classified as false negative tumours, HR=8.79, than for true interval cancers, HR=2.26.


