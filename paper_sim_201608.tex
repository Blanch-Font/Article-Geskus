\documentclass[10pt,a4paper]{article}
\usepackage[latin1]{inputenc}
\usepackage{amsmath}
\usepackage{amsfonts}
\usepackage{amssymb}
\usepackage{fancyref}
\usepackage{graphicx}
\graphicspath{ {/home/jordi/Dropbox/INCA/articulo competing risks} }
\author{Jordi Blanch and Montse Ru\'e}
\title{?}


\begin{document}
\section{Introduction}
Intro sobre screening

Intro sobre models de simulació i screening

Canvi de mentalitat respecte l'anterior "paper".

Objectiu principal: estimació del risc de tenir càncer via MSM o Cox. Quin és millor? Quines hipòtesis i condicions s'han de satisfer? Què passa quan no es compleixen? 
Objectiu secundari: Efecte de l'allargament del follow-up ens els càncers detectats entre dues proves, com si fossin de cribratge en la següent prova.

Estructura: estarà separat en: mètodes, resultats, discussió i conclusions.

\section{Motivating data: the INterval CAncer (INCA) study}
The aim of the INCA study was to assess the determinants of interval breast cancer in women
attending early detection programs in Spain[Domingo and Blanch]. Interval cancers are symptomatic
cancers that appear between two screening exams or during a period after the last screening exam.
The INCA study also compared the characteristics and factors associated with screen-detected and
interval breast cancers. We selected a subset of the INCA dataset, corresponding to 4 radiology 
units of the Catalonia region, the INCA-CAT dataset. 

\paragraph{}Population-based breast cancer screening in Catalonia is offered biennially to all
women aged 50-69 years. Screening mammography has three possible outcomes: 1) negative result
(normal), 2) positive result (abnormal findings requiring further assessments), and 3) early recall
(an intermediate mammogram is performed out of sequence with the screening interval, at 6 or 12
months). Cancers detected at intermediate mammogram were considered screen-detected cancers[Perry].

\paragraph{}A positive result is considered to be a screen-detected tumour if, after further
assessments, there is histopathological confirmation of cancer. Otherwise, the result is considered
false-positive and the woman is invited again after two years. Further assessments can include
non-invasive procedures (magnetic resonance imaging, ultrasonography, additional mammography) and/or
invasive (fine-needle aspiration cytology, core-needle biopsy and open biopsy). 

\paragraph{}Interval cancer was defined as ``a primary breast cancer arising after a negative
screening episode and before the next invitation to screening or within 24 months for women who
reached the upper age limit"[Perry]. We extended the definition until the 30th month, because we
allowed a 6 month margin for women to attend each round. Interval cancers were identified by
merging data from the registers of screening programmes with population-based cancer registries
[Navarro], the regional Minimum Data Set (based on hospital discharges with information on the main
diagnosis) and hospital-based cancer registries.

\paragraph{}The INCA-CAT dataset consists of a retrospective cohort of 96,636 women who underwent
mammography exams between January 1, 2000 and December 31, 2006, and were followed-up until June
30, 2009 for interval cancer assessment. These women underwent a total of 230,742 screening
mammograms. During the study period, 963 cancers were detected in routine screening mammograms, of
which 671 were detected in successive participations, and 313 emerged as interval cancers. Both
invasive and \textit{in situ} breast cancer carcinomas are included. 

\paragraph{Data features}To analysis the INCA-CAT dataset, we assume that
\begin{itemize}
\item Right censoring of clinical stage: All woman will have a clinical breast cancer.
\item Interval censoring of pre-clinical stage: The time to enter in the pre-clinical stage occurs
between to participations.
\item Left truncation: All woman are invited to participate from 50 years, but not all come in
their first invitation or they started the screening before the beginning of the study.
\item Informative censoring of the sojourn time: Women with SDC was treated and the time between
pre-clinical to clinical cancer is not observed. It is a problem, because women with a short
sojourn time are more likely to have an IC and women with a sojourn time longer than 2 years will
always be censored.
\end{itemize}

\section{Methods}
% Especificar com respondrem els dos objectius
% 
% Objectiu principal:
%   - Mètodes: MSM i Cox
%   - Manera: 5 simulacions realitzades
%   - Anàlisi de sensibilitat: amagar informació per reproduir més bé la realitat.
% Objectiu secundari:
%   - Allargem el temps de seguiment fins als 2.5 (o 2) anys per estudiar l'efecte sobre el hazard
%     risc.
Per respondre l'objectiu principal crearem 5 simulacions amb 3 estats amb diferents caracterísiques.
En cada simulació, estimarem un model de Cox i un MSM. Per simular millor la realitat,
aleatoriament amagarem informació de l'entrada a pre-clínic o clínic. Per a l'objectiu secundari,
utilitzarem les mateixes simulacons però simplificarem el model a 2 estats i estudiarem l'efecte de
detectar els IC en el moment que succeixen o a la següent mamografia.

% Mètodes:
%   + MSM: explicació resumida
%   + Cox: explicació resumida
\subsection{Markov multi-state models for panel data}
In this subsection, we summarize the main features of Markov multi-state models (MSM). For a better
explanation, jackson2010.

An MSM model suppose that each individe moves inside of a serie of finite states in continuous
time. An example is the progression of disease, where each individe can progress or recover between
adjacent states or can die (Figure \ref{fig_msm}).

\begin{figure}

\caption{General model for disease progression.}
\label{fig_msm} 
\end{figure}

El moviment entre els estats $r$ i $s$ està definit per ''transition intensities'',
$q_{rs}(t, z(t))$ on $r,s = 1,..., R$. Aquest poden dependre del temps o de les co-variables de
l'individu. La transició entre $(r,s)$ es pot definir com
\[q_{rs}(t, z(t)) = \frac{lim P(S(t + \delta t) = s|S(t) = r)}{\delta t}.\]
Totes les transicions formen la matriu de transicions $Q$ (dimensió $R\times R$) on cada fila suma 0.

% A multi-state model describes how an individual moves between a series of states in continuous
% time. Suppose an individual is in state S(t) at time t. The movement on the discrete state
% space 1, . . . , R is governed by transition intensities q rs (t, z(t)) : r, s = 1, . . . , R.
% These may depend on time t, or, more generally, also on a set of individual-level or
% time-dependent explanatory variables z(t). The intensity represents the instantaneous risk of
% moving from state r to state s 6 = r:
% The q rs form a R×R matrix Q whose rows sum to zero.

This article focuses on multi-state models for panel data, in which the state S i (t) is only
known at a finite series of times t = (t i1 , . . . , t in i ). Fitting multi-state models to panel data
generally relies on the Markov assumption, that future evolution only depends on the current
state. That is, q rs (t, z(t), F t ) is independent of F t , the observation history F t of the process
up to the time preceding t. See, for example, Cox and Miller (1965) for a thorough review of
continuous-time Markov chain theory. In a time-homogeneous Markov model, in which the
q rs are also independent of t, the sojourn time in each state r is exponentially-distributed
with mean −1/q rr . The probability that an individual in state r moves next to state s is
−q rs /q rr .

The Markov model for panel data was first described by Kalbfleisch and Lawless (1985)
and Kay (1986). The likelihood for this basic model, used in msm, is calculated from the
transition probability matrix P (u, t + u). The (r, s) entry of P (u, t + u), p rs (u, t + u), is the
probability of being in state s at a time t + u, given the state at time u is r. P (u, t + u) is
calculated in terms of Q using the Kolmogorov differential equations (Cox and Miller 1965).
If the transition intensity matrix Q is constant over the interval (u, t + u), as in a time-
homogeneous process, then P (u, t + u) = P (t) and the equations are solved by the matrix
exponential of Q scaled by the time interval,
P (t) = Exp(tQ).

The full likelihood is then the product of probabilities of transition between observed states,
over all individuals i and observation times j:
L(Q) =
Y
i
L i =
Y
i,j
L i,j =
Y
p S(t ij )S(t i,j+1 ) (t i,j+1 − t ij ).
(1)
i,j
Each component L i,j is the entry of the transition matrix P (t) at the S(t ij )th row and
S(t i,j+1 )th column, evaluated at t = t i,j+1 − t ij . The likelihood L(Q) is maximized in terms
of log(q rs ) to compute the estimates of q rs , using standard optimization algorithms, as im-
plemented in the optim function in R. Standard errors are computed from the Hessian at the
optimum. Some of these optimization algorithms make u

\paragraph{}

\subsection{Proportional hazard model}

\paragraph{}

% Les simulacions suposen una distribució de Weibull per l'entrada a pre-clínic on l'efecte del FP
% és diferent
%   + 1) no té efecte sobre l'entrada a pre-clínic
%   + 2) FP constant al llarg dels cribratge
%   + 3) FP com a funció step
%   + 4) 3 + amb censores independents
%   + 5) 3 + censures depenen del FP.
% 
% Simulació de la història natural d'una dona amb CM.
%   + Com? Simulació en tres estats: sa -> pre-clínic -> clínic (posa dibuix)
%   + Quantes? 15.000 dones sanes
%   + Simulacions: 1.000 per cada combinació
%   + Entrada:
%     * Sense entrada retardada: totes entren amb 50 anys (t0=0)
%     * Amb entrada retardada: l'entrada seguirà una U(0,15)
%   + Seguiment màxim de 8 anys: 4 participacions més el temps fins a la següent mamografia.
%   + Sortida:
%     * Censura administrativa als 8 anys.
%     * 71.5 anys.
%     * Té un càncer pre-clínic o clínic.
%     * Censura aleatoria: varis escenaris.
% 
% Simualció en 3 estats:
%   + Per a simular les diferents distribucions seguim el ``paper'' de Bender et al.
%   + Paràmetres dels models extrets de http://www.hindawi.com/journals/isrn/2013/750857/
%   + Temps fins a pre-clínic:
%     * Estudiem el temps fins pre-clínic des dels 50 anys.
%     * Temps des de l'entrada a la cohort l'hem descartat.
%     * ``Age is more relevant timescale, since it directly reflects the biological process''
%     * El temps d'entrada a pre-clínic és una censura d'interval.
%     * No coneixem el moment exacta. Sabem que succeix entre l'última prova negativa i la primera
%       positiva.
%     * Distibució de Weibull(scale = 0.002, shape = 0.82).
%     * FP escenaris:
%       1) No tenim FP: Weibull normal
%       2) FP constant en totes les participacions: Formula ...
%       3) FP funció step temps-depenent: Formula Bender
%     * Censura aleatoria:
%       - En els escenaris 1 i 2 no tenim censures aleatories.
%       - En l'escenari 3 hem suposat que no tenim censures aleatories, censura aleatoria global i
%         censura aleatoria depenen del FP (Parametres ``paper'' Roman et al.)
%   + Temps de pre-clinic a clínic:
%     * Estudiarem el sojourn time (temps des de pre-clínic a clínic).
%     * Clock-reset approach (t0=0 (origen quan entra a pre-clínic)).
%     * Suposem censura per la dreta.
%     * ''When the sojourn time is the estimand, such data have been called doubly (interval)
%       censored''
%     * Suposem que el temps de sojourn segueix una distribució exponencial(lambda = 0.4).
%     * Només té censura administrativa al cap de dos anys i mig.
% 
% En cada simulació ens centrarem en dos punts:
%   + Entrada retardada o no:
%     * Tothom entra als 50 anys
%     * Entrada retardada seguint una U(0,15)
%     * Podrem quantificar quin biaix hi ha entre la situació ideal i la real.
%   + Càncers a pre-clínics tractats:
%     * Les dones amb un càncer pre-clínic detectat se les tracta i no es pot observar l'entrada en clínic.
%     * El temps de sojourn té censura informativa en les dones tractades.
%     * Farem dos escenaris un on observem tota la història i l'altra com s'observa en els programes de cribratge.
%     * Revisar literatura.
% 
% També estudiarem el temps fins a pre-clínic, reduint el model a dos estats:
%   + Simulació: sa -> pre-clínic
%   + Censura per la dreta.
%   + Sortida: 
%     * Censura administrativa als 8 anys.
%     * 71.5 anys.
%     * Té un càncer pre-clínic o clínic.
%     * Censura aleatoria: varis escenaris.
%   + Dos escenaris en el cas dels clínics:
%     * Quan detectem el càncer clínic, suposem que és el temps a pre-clínic.
%     * Quan detectem el càncer clínic, allarguem el seguiment fins a la següent prova.

\section{Results}
\subsection{Simulation studies}
Explicar els resultats

Per cada simulació hem variat els següents paràmetres:
  + HR(FP) = 1.5, 2 i 4
  + Parametre de l'exponencial del sojourn time: 0.2, 0.4, 0.8
  + Paràmetres de la Weibull:
    * Shape: 0.60, 0.82, 0.90
    * Scale: 0.03, 0.5, 1

\subsubsection{Simulació dels 3 estats}
\subsubsection{Simulació dels 2 estats}



\subsection{Aplication in INCA-CAT}

\section{Discussion}
  + Simulacions:
    * Biaix entre observat i real
    * Biaix quan tenim entrades retardades
    * Biaix en el model de 2 estats (allargament dels CI fins al següent examen)
    * Què passa quan modifiquem els paràmetres
  + Aplicació a l'INCA:
    * Quins problemes té l'INCA
    * Segons la simulació, quin biaix tenim respecte si ho observéssim tot.

  + Fortaleses i limitacions

\section{Conclusion}

\section{References}

\end{document}