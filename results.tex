\section{Results}
\subsection{Simulation study}
In this section we present the results of the simulation study and try to relate the models
performance to their assumptions and to the specificities of the data. For the complete data
analyses we have assumed that there are three states. We assume that women receive 10 biennial
mammographic exams at the age interval 50-69 years, age at entering screening may or may not have
late entry, and the HR of a false-positive result is 2.

\subsubsection{Transition intensities for the three states multistate model. Complete and observed
               data}
Figures \ref{fig:trans_complete} and \ref{fig:trans_observed} present the simulated (theoretical)
and estimated transition intensities in the age interval 50-69 years for the \textbf{complete} and
the \textbf{observed} datasets, respectively. For complete data, the estimated
$S_0 \rightarrow S_p$ piecewise constant transition rates overlap with the theoretical Weibull
function and the estimated $S_p \rightarrow S_c$ transition rate is unbiased (Figure
\ref{fig:trans_complete}). Late entry does not change these results. With observed data, the
estimated $S_0 \rightarrow S_p$ piecewise constant rates follow the Weibull pattern, similarly to
the complete data scenario, but the estimated transition rate overestimates considerably the
theoretical rate (Figure \ref{fig:trans_observed}).

\subsubsection{Hazard ratio of a FP result, for the three state models. Complete data}
Before interpreting the results it is important to mention that the HR of FP for the transition
$S_p \rightarrow S_c$ only can be estimated when using multistate models. We have simulated the
data assuming that a FP result is associated with the transition $S_0 \rightarrow S_p$ with a
$HR = 2$ and it is not associated with the transition $S_p \rightarrow S_c$ (HR=1). For the
multistate model these values are our theoretical values. For the Cox and discrete time models only
the transitions $S_0 \rightarrow S_p$ and $S_0 \rightarrow S_c$ can be estimated. We have assessed
the performance of these models assuming that the theoretical HR value for both transitions is 2.

\paragraph{}Table \ref{tab:HRFP_complete} shows the performance of the studied models, when
considering \textbf{three states} and \textbf{complete data}, with respect to the estimation of the
theoretical HR of having a FP result in the screening mammogram for the transitions
$S_0 \rightarrow S_p$ and $S_p \rightarrow S_c$. The MS3 model performs very well for both
transitions, either \textit{with or without late entry}, with very low bias and coverage higher
than 95\%.

\paragraph{}The Cox CS model has also good properties when estimating the HR of a FP result for the
$S_0 \rightarrow S_p$ transition. For the $S_0 \rightarrow S_c$ the $\overline{\hat\beta}$ value
slightly underestimates the true $\beta$ value with a bias around 2.5\%, independently of the
presence/absence of \textit{late entry}. Coverage is slightly lower than 95\% for the
$S_0 \rightarrow S_c$ transition with \textit{late entry}.

\paragraph{}The discrete time model with three states (DT3) does not perform as well as the MS3 and
the Cox CS models. In this case the percentage bias of the $\overline{\hat\beta}$ associated to the
$S_0 \rightarrow S_c$ transition approaches 5\% and the coverage of the HR intervals for the
$S_0 \rightarrow S_p$ transition is lower than 80\%, for both scenarios \textit{with/without late
entry}.

\subsubsection{Hazard ratio of a FP result, for the three state models. Observed data}
Table \ref{tab:HRFP_observed3} shows the performance of the studied models for the HR of a FP, when
considering \textbf{three states} and \textbf{observed data}. \textit{With or without late entry},
both the MS3 and the Cox CS models perform well in terms of bias, MSE, and coverage. Instead, the
DT3 model shows considerable bias (around 10\% overestimation) and high MSE when estimating the HR
of a FP result for the $S_0 \rightarrow S_c$ transition. The coverage of the intervals for this HR
is lower than 90\% \textit{without late entry} and near 95\% \textit{with late entry} which
probably is due to wide confidence intervals of the estimated HRs as the high MSE indicates.

\subsubsection{Hazard ratio of a FP result, for the two state models. Observed data}
Table \ref{tab:HRFP_observed2} presents the performance of the studied models for the HR of a FP
result, when considering \textbf{two states} in the \textbf{observed data}. Here it is assumed that
the time to the event of interest is the time when the tumour is detected, either by screening or
clinically. We also have assessed the scenario that, for the clinically detected tumours, assumes
that the time to event is the time at the next screening exam. 

\paragraph{}We observe a good performance of the three studied methods, with a better performance
of the multi-state model with \textit{no late entry} followed by the discrete time (DT2 next
screen) and the Cox model (either exact time or next screen). The DT2 exact time model slightly
overestimates the effect and \textit{with late entry} has lower coverage that the other methods. 

\subsection{Aplication to the INCA-CAT study}
Tables \ref{tab:HR3INCA} and \ref{tab:HR2INCA} present the estimates of the HR of false positive
result when applying the models to the INCA-CAT data. 

\paragraph{}When the studied models were applied to the INCA study and the three states $S_0$,
$S_p$ and $S_c$ were considered, the multi-state and the Cox models provided similar estimates for
the HR of FP on the $S_0 \rightarrow S_p$ transition ($HR \approx 1.77$).  However, the
discrete-time model provided a higher estimate of the HR, $1.93$. For the $S_p \rightarrow S_c$
transition, we obtained a HR near to 1 as in the simulation study. The $S_0 \rightarrow S_c$
transitions estimated with the Cox and the discrete time models provide different values, HR around
$1.92$ and $1.22$, respectively.

\paragraph{}When considering only two states, with no distinction between pre-clinical or
clinically detected cancer, all the three models provide similar results, with the HR of a FP
result varying from $1.68$ to $1.73$, when the time of clinically detected cancer is considered an
exact time. When the time of clinically detected cancer is extended up to the next screening exams,
the three estimated values vary from $1.64$ for the multi-state model to $1.82$ for the
discrete-time model.